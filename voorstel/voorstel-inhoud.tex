% !TeX spellcheck = en_GB
%---------- Inleiding ---------------------------------------------------------
\section{Introduction} % The \section*{} command stops section numbering
\label{sec:Introduction}
\subsection{Situating the subject}
 There has been a strong believe over the last 30 years that quantum computing can and will influence our sophisticated environment more than we think. In case of the mainframe environment it will maybe be the most influenced sector in \emph{computer science}, because of its immense creation of data. Data will become or has already been the driving factor inside our societies, think of how much our daily lives are already controlled by data ( e.g. online shopping, social media etc.). With the usage of mainframes we are able to create a sense of logic in this almost infinite pile of data. Now with \emph{ theoretical} utilisation of quantum computing, data can be searched more thoroughly and faster. \autocite{Grover1996}
 
 If we are able to find and explore quantum applications for our current high-transactional business applications, a new wave of investment in research will open itself up. Which would obviously boost both fields at once. In this paper we will try and find these general applications that can prevail through the use of quantum technology.


%---------- Stand van zaken ---------------------------------------------------

\section{State-of-the-art}
\label{sec:state-of-the-art}
\subsection{Prior knowledge}
Inside the paper a couple of physics specific terms will be utilised. If you are not familiar with basic quantum physics notations and or terms, it would be highly recommended to read one or both of the following papers, ~\autocite{Rieffel1998} or ~\autocite{Shor2000}. For the general quantum notation that are used throughout the field, we refer towards ~\textcite{Dirac1939}. It is also possible to read this paper as an informational piece without the implications of the mathematics and physics surrounding the subject. As previously stated the paper will not be going in depth technologically, because the scope is more focused on exposing the practical usages of quantum computing compared to classical computing or the combination of them both.

\subsection{Recent developments}

As of now Google has claimed to have won the \emph{Quantum Supremacy race} ~\autocite{Google2019} against IBM. They have realised this through the creation of their 54-Qubit quantum computer ( 53 functional qubits), that is able to perform a calculation exponentially faster than a classical system could ever hope to perform. In this case the \emph{Sycamore} ( Quantum processor) was able to perform a calculation within 200 seconds that could only be performed by a classical computer over 10.000 years ( theoretically). Although it most definitely was an experimental calculation that has no real value in the business world, it does however prove the potential of quantum computing. It has been rumoured that IBM will release its counterpart of research in 2020. The fact that these 2 conglomerates are competing so fiercely will only further the technological developments in the realm of quantum mechanics. IBM has not been sitting idly either, they have released a paper regarding quantum algorithms applications. \autocite{IBM2019}



%---------- Methodologie ------------------------------------------------------
\section{Methodology}
\label{sec:Methodology}

While the field of practical quantum computing is still in its infancy, there are a lot of different possible angles to approach the subject with. First of all we will be introducing the guiding principles of quantum computing, as to all start on the same footing. Then we will explore the realistic potential that quantum computing can offer for economic gain, especially for mainframe development. This will mainly be comprised of an extensive literature study that will  set its focus on economic applications of quantum computation and thereby on the mainframe environment.
Finally there will be a demonstration of quantum computation software development through the use of the open-source framework called Qiskit. \autocite{Qiskit} Qiskit is an IBM Python framework that allows its users to simulate quantum circuits to build software, but what makes it really interesting is the fact that IBM allows its users to effectively perform these simulated quantum circuits on existing quantum computers! ( with limited qubits) More precisely, we will be using Qiskit to create a visualisation of the effects of executing different quantum algorithms like Shor's algorithm  ~\autocite{Shor2000} and/ or Grover's algorithm  ~\autocite{Grover1996}. The mere fact that we are able to already create software applications, will only help make computer scientists more interested and creative with possible quantum computational applications.

%---------- Verwachte resultaten ----------------------------------------------
\section{Expected results}
\label{sec:Expected results}

 The paper will try and create a more concrete point of view on the possible features quantum computation can offer.
 Through the analysis of multiple papers, we are hoping to find certain points of contest. These points indicate the highly debated subjects within quantum computing and are therefore extremely valuable. We will be trying to locate and display the business potential within these points of conflict. Currently IBM has created an extremely stable and performant business environment with its mainframe, Z15 and its older versions. Anything that can/ will affect this stable business platform can form a great threat to the way we currently create and process our data. To protect this stable platform, we will be trying to index all the threats and opportunities that come with the introduction of quantum computation in our current computational environment. The second part of the paper will be more software-orientated, where we will be creating a Python application that puts simulation of quantum computation opposed to the real execution on a quantum device. The paper will visualise these probabilistic differences and will try to show the attention points with simulating quantum computers and effectively executing on one. Through the demonstration of quantum computation we are hoping that readers are going to be personally inspired to be creative with the new technology and start developing their first 'Hello World' with Qiskit.

%---------- Verwachte conclusies ----------------------------------------------
\section{Expected conclusions}
\label{sec:Expected conclusions}

We are expecting to \emph{debunk} the more absurd ideas of quantum computing. (e.g. destroying all our encryptions and our society) Concretely, we are going to put the whole subject inside a more realistic 'future' vision. This will hopefully offer readers ideas of possible applications of quantum computation inside their departments ( e.g. Chemistry, Mechanics, Astronomy etc.) Also With software being so readily available for the general public, we expect that quantum computing applications will be created exponentially faster than with the dawn of classical computing 70 years ago. With this train of thought, we are hoping that real economical value can be available within the next decade. Frameworks like Qiskit will be developed further and more powerful quantum computers will be made available towards the public to boost the research in the subject. And with these thoughts we can be certain that interest in quantum computers will only increase.

