% !TeX spellcheck = en_GB
%---------- Inleiding ---------------------------------------------------------
\chapter{Table of contents}
\section{Introduction} % The \section*{} command stops section numbering
\label{sec:introductie}
\subsection{Situating the subject}
 There has been a strong believe over the last 30 years that quantum computing can and will influence our sophisticated environment more than we think. In case of the mainframe environment it will maybe be the most influenced sector in \emph{computer science}, because of its immense creation of data. Data will become or has already been the driving factor inside our societies, think of how much our daily lives are already controlled by data ( e.g. online shopping, social media etc.). With the usage of mainframes we are able to create a sense of logic in this almost infinite pile of data. Now with \emph{ theoretical} utilisation of quantum computing, data can be searched more thoroughly and faster. \autocite{Grover1996}
 
 The main driving factors for technological breakthroughs have always been wars and economics ( e.g. Atomic energy, commercial aircraft, radio etc.). If we are able to find and explore quantum applications for our current high-transactional business applications, a new wave of investment in research will open itself up. Which would obviously boost both fields at once. In this paper we will try and find these general applications that can prevail through the use of quantum technology.

\subsection{Topics}
\begin{itemize}
  \item Security implications with the rise of quantum computing
  \item Efficiently exploring mainframe data using quantum computing
  \item Advantages and disadvantages of combining classical computing with quantum computing
  \item Building quantum software before the creation of the hardware
\end{itemize}

%---------- Stand van zaken ---------------------------------------------------

\section{State-of-the-art}
\label{sec:state-of-the-art}
\subsection{Prior knowledge}
Inside the paper a couple of physics specific terms will be utilised. If you are not familiar with basic quantum physics notations and or terms, it would be highly recommended to read one or both of the following papers, ~\textcite{Rieffel1998} or ~\textcite{Shor2000}. For the general quantum notation that are used throughout the field, we refer towards ~\textcite{Dirac1939}. It is also possible to read this paper as an informational piece without the implications of the mathematics and physics surrounding the subject. As previously stated the paper will not be going in depth technologically, because the scope is more focused on exposing the practical usages of quantum computing compared to classical computing or the combination of both.

\subsection{Recent developments}

As of now Google has claimed to have won the \emph{Quantum Supremacy race} ~\autocite{Google2019} against IBM. They have realised this through the creation of their 54-Qubit quantum computer ( 53 functional qubits), that is able to perform a calculation exponentially faster than an classical system could ever hope to perform. In this case the \emph{Sycamore} ( Quantum processor) was able to perform a calculation within 200 seconds that could only be performed by a classical computer over 10.000 years ( theoretically). Although it most definitely was an experimental calculation that has no real value in the business world, it does prove the potential of quantum computing. IBM, Google's main rival in quantum computing research, has expressed concerns regarding the claim of Google that the task would take 10.000 years on a classical machine. Despite this, Google has still achieved the status of releasing the first paper proving the viability for realistic applications of quantum computing. It has been rumoured that IBM will release its counterpart of research in 2020. The fact that these 2 conglomerates are competing so fiercely will only further the technological developments in the realm of quantum mechanics. IBM has not been sitting idly either, they have released a paper regarding quantum algorithms applications. \autocite{IBM2019}



%---------- Methodologie ------------------------------------------------------
\section{Methodology}
\label{sec:methodologie}

While the field of practical quantum computing is still in its infancy, there are a lot of different possible angles to approach the subject with. First of all we will be introducing the guiding principles of quantum computing, as to all start on the same footing. Then we will explore the realistic potential that quantum computing can offer for economic gain, especially for Mainframe development. This will mainly be comprised of an extensive literature study that will mainly set its focus on economic applications of quantum computation.
Finally there will be a demonstration of quantum computation software development through the use of the open-source framework called Qiskit. \autocite{Qiskit} Qiskit is an IBM Python framework that allows its users to simulate quantum circuits to build software, but what makes it really interesting is the fact that IBM allows its users to effectively perform these simulated quantum circuits on existing quantum computers! ( with limited qubits) The mere fact that we are able to already create software applications, will only help make computer scientists more interested and creative with possible quantum computational applications.

%---------- Verwachte resultaten ----------------------------------------------
\section{Expected results}
\label{sec:verwachte_resultaten}

As to be expected with every new field of study, this one will not be an exception. Meaning that there will be a lot of diverse opinions throughout the literature. Having this fierce arguments between papers only helps the subject move further along, because it challenges everyone to be more critical of all the source material. The paper will try and create a more concrete point of view on the possible features quantum computation can offer. As said before, if we are able to create these business opportunities general interest will only increase. And with the demonstration of quantum computation we are hoping that readers are going to be personally inspired to be creative with the new technology and start developing their first 'Hello World' with Qiskit.

%---------- Verwachte conclusies ----------------------------------------------
\section{Expected conclusions}
\label{sec:verwachte_conclusies}

We are expecting to \emph{debunk} the more absurd ideas of quantum computing. (e.g. destroying all our encryptions and our society) With software being so readily available for the general public, we expect that quantum computing applications will be created exponentially faster than with the dawn of classical computing 70 years ago. With this train of thought, we are hoping that real economical value can be available within the next decade. Frameworks like Qiskit will be developed further and more powerful quantum computers will be made available towards the public to boost the research in the subject.

