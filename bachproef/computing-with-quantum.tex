%%=============================================================================
%% Methodologie
%%=============================================================================

\chapter{\IfLanguageName{dutch}{Methodologie}{Real-world solutions with Quantum}}
\label{ch:computing-with-quantum}

%% TODO: Hoe ben je te werk gegaan? Verdeel je onderzoek in grote fasen, en
%% licht in elke fase toe welke stappen je gevolgd hebt. Verantwoord waarom je
%% op deze manier te werk gegaan bent. Je moet kunnen aantonen dat je de best
%% mogelijke manier toegepast hebt om een antwoord te vinden op de
%% onderzoeksvraag.

Once you start looking into quantum theory and everything it could possibly do for your scientific project, you would find yourself in one of the deepest rabbit holes you could ever possibly find. The true value of quantum research is how we can actually use it for real-life solutions. One could easily imagine that being able to simulate an exact medicine within a couple of days instead of the many months it takes at the moment, would save a numerous amount of lives. So keeping this same train of thought throughout, it is of great importance that we actually focus our attention what current developments could possibly mean for existing projects and research.

\subsection{Quantum computing and traditional computing}

Quantum computing is and will never be the sole solution to a problem. This new form of computing is made to be an addition to points where classical computing fails, e.g. searching through an extremely large dataset without having a clear index within a polynomial time frame such as in \textcite{Terhal1998}. Quantum computing also will have its limits it takes a lot longer to actually set up you computation than it would on a regular machine, but it could be able to solve a couple of non polynomial problems we are currently facing in computer science like factorization. Some problems have been left NP-complete even with quantum like this paper has tried, \textcite{Wang2007}.

Classical computing is still great at organising stuff and performing parallel actions on your device, but with the help of quantum computing we would be able to shift the heavy long term calculations over to devices especially made for long term and hard calculations like a quantum computer. Calculating a machine learning model or performing an accurate simulation of an economic sector could be exponentially reduced in time, which would in turn return the value of these calculations to business men in a much faster way and with that would be even more valuable to them if the information is gathered in a proper time frame. \autocite{Schuld2015} \autocite{Troyer2005}

\subsection{Quantum computing and the mainframe}

First of all we need to clarify what a mainframe is and what its main use is in our current business environments. A mainframe is a type of supercomputer that is different from other supercomputers because it is not specialised in solving 1 really hard problem, like simulations or factorisation, it is specialised to have the highest possible throughput for smaller calculations. The mainframe is widely used within the banking, production and logistical sector as it offers the most reliable way of managing your data that is generated by a certain business practice. To clarify let us look at an example where a mainframe computer like the IBM Z15 shines. When millions of users throughout the world want to buy their flight tickets towards France around the end of April, a huge bottleneck is created at the end point of the booking system of the particular airport. A mainframe handles these types of transactions to make sure every single booking will come through with the correct data and if the data is corrupted along the way, the mainframe is able to spot out these irregularities and discard this data so that the user received a proper notification as soon as possible. So look at a mainframe computer as a really good processor of input and output.

So now that you are able to view what role the mainframe plays, we can more clearly look at how quantum computers could offer major benefits as a complementary service for solving the harder problems just like a super computer works with the mainframe in much the same way. Nowadays all the data generated from the billions of transactions from the mainframe are preserved so that afterwards a supercomputer would be able to process all this information inside a reasonable time frame to get as most as possible business value out of it. If the quantum computer would be able to help process this data exponentially faster, the business value of this data would also exponentially increase.

IBM has released the new mainframe in 2019, Z15, with a broad future perspective, because as one of the top researchers in quantum technology they have a clear image of how a quantum computer could influence themselves and others within their sector.
They are emphasising on 2 very different aspects to make sure their devices are the most likely to take the biggest market share, modernisation and security. 
With modernisation IBM is trying the incorporate the mainframe in as much areas as possible to keep on attracting new developers so that their devices don't fall behind. And with this modernisation a lot of opportunities are opening up to connect different departments such as quantum research with data engineering etc. 

Also emphasising on creating new security measures which focusses more on digital signing than the current RSA factorisation algorithm could secure the mainframe security status indefinitely. Quantum would in the future indeed be able to break these RSA based algorithms and that is why data-security has become such a high importance area at the moment for everyone in computer science.
