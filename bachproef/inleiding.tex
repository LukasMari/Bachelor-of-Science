%%=============================================================================
%% Inleiding
%%=============================================================================

\chapter{\IfLanguageName{dutch}{Inleiding}{Introduction}}
\label{ch:inleiding}

Why does everyone suddenly jump on the subject of quantum computing \textbf{(QC)} and why would it be of any concern at this point in time? Well, we are rapidly reaching the limits of how small we are able to create the transistors on a chip, Moore's Law may very well be about to end. Currently we are able to create transistors so small that they themselves start being influenced by the quantum world which would undermine the whole point of building smaller and smaller components that are faster than its predecessors. \autocite{Hartnett2019}

The quantum field itself is also rapidly expanding due to the practical executions of QC, much in the same way data science has been expanding for the last 2 decades. Only 20 years ago data was something useful to have in a business to gain a possible edge over opponents, now data is the lifeblood for many of those companies. Data is what drives research, competitive advantage and various innovations. QC could offer our way of data handling and processing a surprising speed boost and expansion into regions we just were not able to understand due to its quantum nature e.g. in sectors like chemistry, astronomy, physics...  And this is exactly why the mainframe environment could tie so nicely into the research towards a classical and quantum computational combined environment. Mainframes drive the big enterprises who in turn drive the smaller ones that our societies are built upon. If QC could aid these big enterprises they would in turn let this information flow through into the lower sectors of our global economy. All this is why in this paper we will try and explain why we should start keeping track of QC in the very same way everyone suddenly started keeping track of data research with classical computing.~\autocite{Google2019} ~\autocite{IBM2019}


\section{\IfLanguageName{dutch}{Probleemstelling}{Problem Statement}}
\label{sec:probleemstelling}

What can QC actually solve this very moment, is a question every interested enterprise is trying to figure out first. The field has shown much promise even in this very early state with the limited amount of computational resources much promise. Inside the fields of data processing there is a clear trend that as we further develop quantum computational expertise the possible business impacts are generated exponentially. Many big enterprises have finally figured out the most lucrative and easy-to-apply ways of capturing important data that could have business value. Now the actual issue that most definitely is worth addressing, is that the processing of the shear amount of data has become unbearable in realistic time schemes.\autocite{Rieffel1998}

A business would need all these data results as soon as possible to gain the edge over competitors and industry leaders. Quantum could in theory exponentially aid classical computers with the processing of this large amount of data. Mainframes especially are able to generate so much I/O with all sorts of data e.g. credit card spending, production analysis, transport optimization, that the mainframe together with QC could very well become the power couple of the 21st century. With platforms such as Qiskit everyone is able to contribute towards quantum research even within a mainframe minded environment.  ~\autocite{Qiskit} and ~\autocite{Cirq}

\section{\IfLanguageName{dutch}{Onderzoeksvraag}{Research question}}
\label{sec:onderzoeksvraag}

"How will QC affect the mainframe environment and its applications?" can be a really useful question to solve because it would allow the highly expertised environment of the mainframe to be able to think of possible applications of the quantum research with their own systems. In this moment practical QC has come such a long way that this question could possibly give added business value to the mainframe industry and all of their users. Exploring this domain can provide valuable insights in all different kinds of sectors that make use of the mainframe's high-data capacities.

\section{\IfLanguageName{dutch}{Onderzoeksdoelstelling}{Research objective}}
\label{sec:onderzoeksdoelstelling}

This paper is designed to allow individuals that are interested in QC and general computer science  to get a better grasp of the real business impacts of the quantum realm. We expect that we can evaluate the potential of QC inside a business, but also we want to find actual value and applications inside the sector as a whole. Following up with the real practical showcase of these new quantum technologies, we will be using a Python framework that is freely available for anyone to download and use. The framework is called \href{https://qiskit.org/}{\textit{Qiskit}} and is at the moment the only framework that allows the connection to real quantum devices. This connection is free of charge,  you are able to create quantum circuits and test them out on simulators of the framework  itself to then follow these circuits up by sending them off to a quantum device and get back the results. By being able to look at real executions you instantaneously receive a better grasp of the whole aspect of certain quantum phenomena.

\section{\IfLanguageName{dutch}{Opzet van deze bachelorproef}{Structure of this bachelor thesis}}
\label{sec:opzet-bachelorproef}

This paper will consist of the following chapters:

In chapter ~\ref{ch:quantum-essentials}, we will introduce the very basic usage of QC to make sure every reader is able to understand the basic necessary principles.

In chapter ~\ref{ch:computing-with-quantum}, we will explain how classical and QC could offer a valuable partnership in their effort to speed up all research. 

In chapter ~\ref{ch:practical}, we will finally show the real usages of quantum algorithms of the future through simulations or even executions on real devices with Qiskit. The algorithm that we will use will be an adaptation of Grover's algorithm for a unstructured search, more specifically an algorithm that is able to solve the 3-SAT problem in an quantum manner.

In chapter~\ref{ch:conclusie}, there will be a discussion where we take in the results of the practical compartment of this paper. finally we will take a critical look at how QC has its benefits but also its disadvantages. 