%%=============================================================================
%% Inleiding
%%=============================================================================

\chapter{\IfLanguageName{dutch}{Inleiding}{Introduction}}
\label{ch:inleiding}

Why does everyone suddenly jump on the subject of quantum computing \textbf{(QC)} and why would it concern anyone at this point in time? Well we are rapidly reaching the limits of how small we are able to create the transistors on a chip, Moore's Law may very well be about to end. Currently we are able to create transistors so small that they themselves start being influenced in a quantum way which would undermine the whole point of building smaller and smaller components that are faster than its predecessors. The quantum field itself is also rapidly expanding due to the practical executions of QC ~\textcite{Google2019}, ~\textcite{IBM2019}, much in the same way data science has been expanding for the last 2 decades. Only 20 years ago data was something nice to have in a business to gain a possible edge over opponents, now data is the lifeblood for many of those companies. Data is what drives research, competitive advantage and various innovations. QC could offer our way of data handling and processing a surprising speed boost and expansion into regions we just were not able to even understand due to its quantum nature e.g. Chemistry, Astronomy, Physics...  And this is exactly why the mainframe environment could tie in so nicely into the research towards a classical and quantum computational environment. Mainframes drive the big enterprises who in turn drive the smaller ones that our societies are built upon, if QC can aid the big enterprises they would in turn let this information flow into the lower sectors of our global economy. All this is why the paper will try and expose why we should start caring about QC in the very same everyone suddenly started caring about data research with classical computing.


\section{\IfLanguageName{dutch}{Probleemstelling}{Problem Statement}}
\label{sec:probleemstelling}

What can QC actually solve as of this very moment, is a question every enterprise is trying to figure out first. The field has shown even in this very early state with the limited amount of computational resources much promise. Inside the fields of data processing there is a clear trend that as we further develop quantum computational expertise the possible business impacts are generated exponentially. Many big enterprises have finally figured out the most lucrative and easy-to-apply ways of capturing important data that could have business value. Now the actual issue that most definitely is worth addressing, is that the processing of the shear amount of data has become unbearable in realistic time schemes. Business needs all those data results as soon as possible to gain the edge over competitors and industry leaders, quantum could exponentially aid classical computers with the processing of this data. Mainframes especially are able to generate so much I/O with all sorts of data e.g. credit card spending, production analysis, transport optimization, that the mainframe together with QC could very well become the power couple of the 21st century. With platforms as Qiskit ~\textcite{Qiskit} and ~\textcite{Cirq} everyone is able to contribute towards quantum research even within a mainframe minded environment.

\section{\IfLanguageName{dutch}{Onderzoeksvraag}{Research question}}
\label{sec:onderzoeksvraag}

The question of "How will QC affect the mainframe environment and its applications?" can be a really useful question to solve because it would allow the highly expertised environment of the mainframe to be able to think of possible applications of the quantum research with their mainframe systems. In this moment of time practical QC has come such a long way that this question could possibly give business value to the mainframe industry and all of their users. Exploring this domain can provide valuable insights in all different kinds of sectors that make use of the mainframe high-data capabilities.

\section{\IfLanguageName{dutch}{Onderzoeksdoelstelling}{Research objective}}
\label{sec:onderzoeksdoelstelling}

The paper is designed to allow individuals that are interested in QC and are interested in research to get a better grasp on the real business impacts of the quantum realm. We expect that we can evaluate the real value of QC inside a business, but also we want to find actual value and applications inside the sector as a whole. We also would like to show the practical execution of quantum algorithms and their advantages using Qiskit, which is a python-based framework that is open source to anyone interested in computer science and QC.

\section{\IfLanguageName{dutch}{Opzet van deze bachelorproef}{Structure of this bachelor thesis}}
\label{sec:opzet-bachelorproef}

% Het is gebruikelijk aan het einde van de inleiding een overzicht te
% geven van de opbouw van de rest van de tekst. Deze sectie bevat al een aanzet
% die je kan aanvullen/aanpassen in functie van je eigen tekst.

The paper will consist of the following chapters:

In chapter ~\ref{ch:quantum-essentials}, we will introduce the very basic usage of QC to make sure every reader is able to understand the basic necessary principles to understand this paper.

In chapter ~\ref{ch:computing-with-quantum}, we will expose how classical and QC could offer a valuable partnership in their effort to speed up all research. 

In chapter ~\ref{ch:practical}, we will finally show the real usages of quantum algorithms of the future through simulations or even executions on real devices with Qiskit. This chapter will be designed for computer scientist that want to really understand the technology and want to learn and maybe even contribute themselves towards the many open source options out there surrounding QC.

In chapter~\ref{ch:conclusie}, there will be an extensive discussion where we take in the results of the practical compartment of this paper. Furthermore we would still like to take a critical look at how QC has its disadvantages and maybe even its flaws. 