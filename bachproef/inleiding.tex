%%=============================================================================
%% Inleiding
%%=============================================================================

\chapter{\IfLanguageName{dutch}{Inleiding}{Introduction}}
\label{ch:inleiding}

Why does everyone suddenly jump on the subject of quantum computing and why would it concern anyone at this point in time? The field is rapidly expanding due to the practical executions of quantum computing ~\textcite{Google2019}, ~\textcite{IBM2019}, much in the same way data science has been expanding for the last 2 decades. Only 20 years ago data was something nice to have in a business to gain a possible edge over opponents, now data is the lifeblood for many of those companies. Data is what drives research, competitive advantage and various innovations. Quantum computing could offer our way of data handling and processing a surprising speed boost and expansion into regions we just were not able to even understand due to its quantum nature e.g. Chemistry, Astronomy, Physics...  And this is exactly why the mainframe environment could tie in so nicely into the research towards a classical and quantum computational environment. Mainframes drive the big enterprises who in turn drive the smaller ones that our societies are built upon, if quantum computing can aid the big enterprises they would in turn let this information flow into the lower sectors of our global economy. All this is why the paper will try and expose why we should start caring about quantum computing in the very same everyone suddenly started caring about data research with classical computing.


\section{\IfLanguageName{dutch}{Probleemstelling}{Problem Statement}}
\label{sec:probleemstelling}

What can quantum computing actually solve as of this very moment, is a question every enterprise is trying to figure out first. The field has shown even in this very early state with the limited amount of computational resources much promise. Inside the fields of data processing there is a clear trend that as we further develop quantum computational expertise the possible business impacts are generated exponentially. Many big enterprises have finally figured out the most lucrative and easy-to-apply ways of capturing important data that could have business value. Now the actual issue that most definitely is worth addressing, is that the processing of the shear amount of data has become unbearable in realistic time schemes. Business needs all those data results as soon as possible to gain the edge over competitors and industry leaders, quantum could exponentially aid classical computers with the processing of this data. Mainframes especially are able to generate so much I/O with all sorts of data e.g. credit card spending, production analysis, transport optimization, that the mainframe together with quantum computing could very well become the power couple of the 21st century. With platforms as Qiskit ~\textcite{Qiskit} and ~\textcite{Cirq} everyone is able to contribute towards quantum research even within a mainframe minded environment.

\section{\IfLanguageName{dutch}{Onderzoeksvraag}{Research question}}
\label{sec:onderzoeksvraag}

The question of "How will quantum computing affect the mainframe environment and its applications?" can be a really useful question to solve because it would allow the highly expertised environment of the mainframe to be able to think of possible applications of the quantum research with their mainframe systems. In this moment of time practical quantum computing has come such a long way that this question could possibly give business value to the mainframe industry and all of their users. Exploring this domain can provide valuable insights in all different kinds of sectors that make use of the mainframe high-data capabilities.

\section{\IfLanguageName{dutch}{Onderzoeksdoelstelling}{Research objective}}
\label{sec:onderzoeksdoelstelling}

The paper is designed to allow individuals that are interested in quantum computing and are active in the mainframe environment to get a better grasp on the real business impacts of the quantum realm. We expect that we can debunk most of the absurd ideas surrounding quantum, but also we expect to find actual business value and applications inside these 2 sectors. We also would like to show the practical execution of quantum algorithms and their advantages with Qiskit, which is a python-based framework that is open source to anyone interested in computer science and quantum computing.

\section{\IfLanguageName{dutch}{Opzet van deze bachelorproef}{Structure of this bachelor thesis}}
\label{sec:opzet-bachelorproef}

% Het is gebruikelijk aan het einde van de inleiding een overzicht te
% geven van de opbouw van de rest van de tekst. Deze sectie bevat al een aanzet
% die je kan aanvullen/aanpassen in functie van je eigen tekst.

The paper will consist of the following chapters:

In chapter ~\ref{ch:quantum-essentials}, we will introduce the very basic usage of quantum computing to make sure every reader is able to understand the basic necessary principles to understand this paper.

In chapter ~\ref{ch:computing-with-quantum}, we will expose how classical and quantum computing could offer a valuable partnership in their effort to speed up all research. 

In chapter ~\ref{ch:beneficiary-sectors}, the focus will be more towards economical advantages that quantum could offer the big enterprises with mainframes.

In chapter ~\ref{ch:creating-the-future}, will lay out a realistic future perspective that executives should keep in the back of their mind to make sure they do not fall behind in research or usage of new technologies.

In chapter ~\ref{ch:practical}, we will finally show the real usages of quantum algorithms of the future through simulations or even executions on real devices with Qiskit. This chapter will be designed for computer scientist that want to really understand the technology and want to learn and maybe even contribute themselves towards the many open source options out there surrounding quantum computing.

In Hoofdstuk~\ref{ch:conclusie}, tenslotte, wordt de conclusie gegeven en een antwoord geformuleerd op de onderzoeksvragen. Daarbij wordt ook een aanzet gegeven voor toekomstig onderzoek binnen dit domein.