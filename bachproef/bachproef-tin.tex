%===============================================================================
% LaTeX sjabloon voor de bachelorproef toegepaste informatica aan HOGENT
% Meer info op https://github.com/HoGentTIN/bachproef-latex-sjabloon
%===============================================================================

\documentclass{bachproef-tin}

\usepackage{hogent-thesis-titlepage} % Titelpagina conform aan HOGENT huisstijl

%%---------- Documenteigenschappen ---------------------------------------------
% TODO: Vul dit aan met je eigen info:

% De titel van het rapport/bachelorproef
\title{How will quantum computing affect the mainframe environment and its applications?}

% Je eigen naam
\author{Lukas Marivoet}

% De naam van je promotor (lector van de opleiding)
\promotor{Martijn Saelens}

% De naam van je co-promotor. Als je promotor ook je opdrachtgever is en je
% dus ook inhoudelijk begeleidt (en enkel dan!), mag je dit leeg laten.
\copromotor{Francis Harkins}

% Indien je bachelorproef in opdracht van/in samenwerking met een bedrijf of
% externe organisatie geschreven is, geef je hier de naam. Zoniet laat je dit
% zoals het is.
\instelling{---}

% Academiejaar
\academiejaar{2019-2020}

% Examenperiode
%  - 1e semester = 1e examenperiode => 1
%  - 2e semester = 2e examenperiode => 2
%  - tweede zit  = 3e examenperiode => 3
\examenperiode{2}

%===============================================================================
% Inhoud document
%===============================================================================

\begin{document}

%---------- Taalselectie -------------------------------------------------------
% Als je je bachelorproef in het Engels schrijft, haal dan onderstaande regel
% uit commentaar. Let op: de tekst op de voorkaft blijft in het Nederlands, en
% dat is ook de bedoeling!

\selectlanguage{english}

%---------- Titelblad ----------------------------------------------------------
\inserttitlepage

%---------- Samenvatting, voorwoord --------------------------------------------
\usechapterimagefalse
%%=============================================================================
%% Voorwoord
%%=============================================================================

\chapter*{\IfLanguageName{dutch}{Woord vooraf}{Preface}}
\label{ch:voorwoord}

%% TODO:
%% Het voorwoord is het enige deel van de bachelorproef waar je vanuit je
%% eigen standpunt (``ik-vorm'') mag schrijven. Je kan hier bv. motiveren
%% waarom jij het onderwerp wil bespreken.
%% Vergeet ook niet te bedanken wie je geholpen/gesteund/... heeft

Why did I choose to explore the quantum realm without any specific pre-knowledge/ education? Quantum is being transformed to a real buzzword much like data science used to be. The field has been opened up from highly specialised academics to an open source community willing to teach outsiders from the very start.

Also the mere fact that the field of quantum computing is developing to a profitable and sustainable business models so rapidly has astonished me from my very first contacts with the environment.

 I would like to thank Frank Harkins from IBM for always being available to have a discussion about quantum computing and how it will influence our societies and even our very nature of problem-solvings. 
 
 But most of all I would like to congratulate the research environment as a around quantum computing on how accepting and supportive they are in all interested parties. It has become as comparable to learning a new sector inside computer science. In the next decade this will only further increase to where quantum computing becomes an essential part of solving anything in a fair time scheme.
 
 So that is exactly why this paper should serve as the starting tool for computer scientist willing to expand their skill sets far outside the skill sets that is to be expected of a single computer scientist. 

%%=============================================================================
%% Samenvatting
%%=============================================================================

% TODO: De "abstract" of samenvatting is een kernachtige (~ 1 blz. voor een
% thesis) synthese van het document.
%
% Deze aspecten moeten zeker aan bod komen:
% - Context: waarom is dit werk belangrijk?
% - Nood: waarom moest dit onderzocht worden?
% - Taak: wat heb je precies gedaan?
% - Object: wat staat in dit document geschreven?
% - Resultaat: wat was het resultaat?
% - Conclusie: wat is/zijn de belangrijkste conclusie(s)?
% - Perspectief: blijven er nog vragen open die in de toekomst nog kunnen
%    onderzocht worden? Wat is een mogelijk vervolg voor jouw onderzoek?
%
% LET OP! Een samenvatting is GEEN voorwoord!

%%---------- Nederlandse samenvatting -----------------------------------------
%
% TODO: Als je je bachelorproef in het Engels schrijft, moet je eerst een
% Nederlandse samenvatting invoegen. Haal daarvoor onderstaande code uit
% commentaar.
% Wie zijn bachelorproef in het Nederlands schrijft, kan dit negeren, de inhoud
% wordt niet in het document ingevoegd.

\IfLanguageName{english}{
\selectlanguage{dutch}
\chapter*{Samenvatting}

Zoals het onderzoek aantoont is het onderzoeksgebied van kwantum computers nog in zijn beginjaren en moeten we kritisch blijven ten opzichte van elke nieuwe uitgave in verband met nieuw onderzoek naar kwantum computers. Echter is het ook belangrijk dat we buiten het puur theoretische deel ook effectief op zoek gaan naar de praktische toepassingen en/ of inzichten in onze huidige processen met evt. de toepassing van kwantum verwerking in deze bestaande processen. 

In het onderzoek proberen we een duidelijk beeld weer te geven aan de lezer, zodat hij/ zij zelfstandig kan nadenken over toepassingen en/ of zelf toevoegingen kan maken aan de vele open source gemeenschappen op Github. Dit hebben we proberen te bereiken door enkele praktische weergaves te maken met de hulp van het Python-framework Qiskit over de uitvoering op een kwantum systeem. De resultaten wijzen inderdaad erop dat we bestaande problemen ook kunnen oplossen met kwantum algoritmes, maar zoals te zien aan de werkelijke uitvoeringen op de echte kwantum computers van IBM is het moeilijk om deze nieuwe technologieën al meteen te gebruiken in bestaande productiesystemen. 

In de paper is er ook een focus gelegd op de mogelijke samenwerking van de mainframe in het verhaal over kwantum computers. De verwachtingen zijn al enorm over de samenwerking tussen onze huidige supercomputers en een kwantum systeem, maar dit betekent dus ook dat er een enorm potentieel bestaat tussen een mainframe machine en een kwantum systeem. De potentiële winst van kwantum wordt voorzien in enorme versnellingen van data verwerkingen en dat is nu exact waar een mainframe machine zo krachtig in is, het genereren van enorme hoeveelheden data.

Er ligt ook een sterke nadruk op het tonen van alle kanten van kwantum computers om er zeker van te zijn dat de lezer een volledig beeld heeft van het complete onderwerp.



\selectlanguage{english}
}{}

%%---------- Samenvatting -----------------------------------------------------
% De samenvatting in de hoofdtaal van het document

\chapter*{\IfLanguageName{dutch}{Samenvatting}{Abstract}}

The field of quantum computers is still in its infancy and we must remain critical of any new publication related to new insights. However, it is also important that beyond the purely theoretical part, we also effectively look for practical applications and/or insights into our current processes with the possible application of quantum processing in them. 

In the research we try to present a more clear picture to the reader, so that he or she can independently think about applications and/or make additions to the many open source communities on Github. We have tried to achieve this by making some practical showcases with the help of the Python framework Qiskit for quantum computing. The results indeed point to the possible execution of quantum algorithms on already existing computer science problems. But as can be seen from the actual results on IBM's real quantum computers, it is difficult to use these new technologies in existing production systems at this point in time. 

In the paper there is also a focus on the cooperation of the mainframe in the story of quantum computers. The expectations are huge about the cooperation between our current supercomputers and a quantum system, but this also means that there is a huge potential between a mainframe machine and a quantum system. The benefits of quantum are mostly expected in huge accelerations of data processing and that is exactly what a mainframe machine is so powerful at, generating huge amounts of data.

There is also a strong emphasis on showing all sides of quantum computing as to make sure the reader has a full picture of the whole field.



%---------- Inhoudstafel -------------------------------------------------------
\pagestyle{empty} % Geen hoofding
\tableofcontents  % Voeg de inhoudstafel toe
\cleardoublepage  % Zorg dat volgende hoofstuk op een oneven pagina begint
\pagestyle{fancy} % Zet hoofding opnieuw aan

%---------- Lijst figuren, afkortingen, ... ------------------------------------

% Indien gewenst kan je hier een lijst van figuren/tabellen opgeven. Geef in
% dat geval je figuren/tabellen altijd een korte beschrijving:
%
%  \caption[korte beschrijving]{uitgebreide beschrijving}
%
% De korte beschrijving wordt gebruikt voor deze lijst, de uitgebreide staat bij
% de figuur of tabel zelf.

\listoffigures
\listoftables

% Als je een lijst van afkortingen of termen wil toevoegen, dan hoort die
% hier thuis. Gebruik bijvoorbeeld de ``glossaries'' package.
% https://www.overleaf.com/learn/latex/Glossaries

%---------- Kern ---------------------------------------------------------------

% De eerste hoofdstukken van een bachelorproef zijn meestal een inleiding op
% het onderwerp, literatuurstudie en verantwoording methodologie.
% Aarzel niet om een meer beschrijvende titel aan deze hoofstukken te geven of
% om bijvoorbeeld de inleiding en/of stand van zaken over meerdere hoofdstukken
% te verspreiden!

%%=============================================================================
%% Inleiding
%%=============================================================================

\chapter{\IfLanguageName{dutch}{Inleiding}{Introduction}}
\label{ch:inleiding}

Why does everyone suddenly jump on the subject of quantum computing \textbf{(QC)} and why would it concern anyone at this point in time? Well we are rapidly reaching the limits of how small we are able to create the transistors on a chip, Moore's Law may very well be about to end. Currently we are able to create transistors so small that they themselves start being influenced in a quantum way which would undermine the whole point of building smaller and smaller components that are faster than its predecessors. The quantum field itself is also rapidly expanding due to the practical executions of QC ~\textcite{Google2019}, ~\textcite{IBM2019}, much in the same way data science has been expanding for the last 2 decades. Only 20 years ago data was something nice to have in a business to gain a possible edge over opponents, now data is the lifeblood for many of those companies. Data is what drives research, competitive advantage and various innovations. QC could offer our way of data handling and processing a surprising speed boost and expansion into regions we just were not able to even understand due to its quantum nature e.g. Chemistry, Astronomy, Physics...  And this is exactly why the mainframe environment could tie in so nicely into the research towards a classical and quantum computational environment. Mainframes drive the big enterprises who in turn drive the smaller ones that our societies are built upon, if QC can aid the big enterprises they would in turn let this information flow into the lower sectors of our global economy. All this is why the paper will try and expose why we should start caring about QC in the very same everyone suddenly started caring about data research with classical computing.


\section{\IfLanguageName{dutch}{Probleemstelling}{Problem Statement}}
\label{sec:probleemstelling}

What can QC actually solve as of this very moment, is a question every enterprise is trying to figure out first. The field has shown even in this very early state with the limited amount of computational resources much promise. Inside the fields of data processing there is a clear trend that as we further develop quantum computational expertise the possible business impacts are generated exponentially. Many big enterprises have finally figured out the most lucrative and easy-to-apply ways of capturing important data that could have business value. Now the actual issue that most definitely is worth addressing, is that the processing of the shear amount of data has become unbearable in realistic time schemes. Business needs all those data results as soon as possible to gain the edge over competitors and industry leaders, quantum could exponentially aid classical computers with the processing of this data. Mainframes especially are able to generate so much I/O with all sorts of data e.g. credit card spending, production analysis, transport optimization, that the mainframe together with QC could very well become the power couple of the 21st century. With platforms as Qiskit ~\textcite{Qiskit} and ~\textcite{Cirq} everyone is able to contribute towards quantum research even within a mainframe minded environment.

\section{\IfLanguageName{dutch}{Onderzoeksvraag}{Research question}}
\label{sec:onderzoeksvraag}

The question of "How will QC affect the mainframe environment and its applications?" can be a really useful question to solve because it would allow the highly expertised environment of the mainframe to be able to think of possible applications of the quantum research with their mainframe systems. In this moment of time practical QC has come such a long way that this question could possibly give business value to the mainframe industry and all of their users. Exploring this domain can provide valuable insights in all different kinds of sectors that make use of the mainframe high-data capabilities.

\section{\IfLanguageName{dutch}{Onderzoeksdoelstelling}{Research objective}}
\label{sec:onderzoeksdoelstelling}

The paper is designed to allow individuals that are interested in QC and are interested in research to get a better grasp on the real business impacts of the quantum realm. We expect that we can evaluate the real value of QC inside a business, but also we want to find actual value and applications inside the sector as a whole. We also would like to show the practical execution of quantum algorithms and their advantages using Qiskit, which is a python-based framework that is open source to anyone interested in computer science and QC.

\section{\IfLanguageName{dutch}{Opzet van deze bachelorproef}{Structure of this bachelor thesis}}
\label{sec:opzet-bachelorproef}

% Het is gebruikelijk aan het einde van de inleiding een overzicht te
% geven van de opbouw van de rest van de tekst. Deze sectie bevat al een aanzet
% die je kan aanvullen/aanpassen in functie van je eigen tekst.

The paper will consist of the following chapters:

In chapter ~\ref{ch:quantum-essentials}, we will introduce the very basic usage of QC to make sure every reader is able to understand the basic necessary principles to understand this paper.

In chapter ~\ref{ch:computing-with-quantum}, we will expose how classical and QC could offer a valuable partnership in their effort to speed up all research. 

In chapter ~\ref{ch:practical}, we will finally show the real usages of quantum algorithms of the future through simulations or even executions on real devices with Qiskit. This chapter will be designed for computer scientist that want to really understand the technology and want to learn and maybe even contribute themselves towards the many open source options out there surrounding QC.

In chapter~\ref{ch:conclusie}, there will be an extensive discussion where we take in the results of the practical compartment of this paper. Furthermore we would still like to take a critical look at how QC has its disadvantages and maybe even its flaws. 
\chapter{\IfLanguageName{dutch}{Stand van zaken}{Quantum Essentials}}
\label{ch:quantum-essentials}

To make sure everyone starts from the same baseline to understand the full potential of this paper, we will introduce a few of the basic quantum principles. This paper is not targeting these specific principles but does use them to explain different practical consequences of the use of them within quantum computation. If there is any further interest regarding these principles, we would refer you to the following papers, ~\textcite{Rieffel1998} and ~\textcite{Shor2000}.

\subsection{The Qubit and its classical opponent}

The foundation of any quantum related paper is and will always be the qubit. A qubit is just like a classical computing bit the foundational unit of its computer. Whilst a bit can either be on or off, a qubit has a certain statistical measurement to it. To be able to program on a quantum computer you need to think of the issue of computing an equation in a completely different way. 

During the execution of your program you are simply not able to look at the intermediary results as this would affect the final result, which would make the whole computation worthless. This means that debugging and looking at variables whilst you are executing a piece of code simply is not possible, which in turn makes writing actual code for a quantum computer a lot more difficult. 

To comprehend the nature of a qubit we need to understand that representing a qubit is only possible in a complex field, which shows of a certain amplitude of the state of the qubit in a point in time. \textit{Felix Bloch} was the individual that came up with the Bloch sphere that we currently use to clearly represent what a qubit is at a certain point in time. 

\begin{figure}[h]
\centering
\includegraphics[scale = 0.75]{../Demonstration/img/Quantum_essentials_1.PNG}
\caption{A Bloch sphere representation of 1 qubit in the |1> state. The Bloch sphere clearly indicates that the state of a qubit has a certain probabilistic aspect to it.}
\end{figure}




superposition

entanglement

teleportation

dense coding





%%=============================================================================
%% Methodologie
%%=============================================================================

\chapter{\IfLanguageName{dutch}{Methodologie}{Real-world solutions with Quantum}}
\label{ch:computing-with-quantum}

Once we started looking into quantum theory and everything it could possibly encompass for our scientific project, we found ourselves in one of the deepest rabbit holes we could have possibly found. The true value of quantum research is how we can actually use it for real-life solutions. One could easily imagine that being able to simulate an exact medicine within a couple of days instead of the many months it takes at the moment, would save a numerous amount of lives. So keeping this same train of thought throughout, it is of great importance that we actually focus our attention on what current developments could possibly mean for existing projects and research.

\subsection{Quantum computing and traditional computing}

QC is and will never be the sole solution to a problem. This new form of computing is made to be an addition to points where classical computing fails, e.g. searching through an extremely large dataset without having a clear index within a polynomial time frame such as in \textcite{Terhal1998}. QC also has its limits as it takes a lot longer to actually set up your computation than it would take on a regular machine. But it could be able to solve a couple of non polynomial problems we are currently facing in computer science like factorization. 
Some problems have been left NP-complete (non-polynomial) even with quantum attempts like this paper has tried, \textcite{Wang2007}. Quantum computing must not be looked at as the single solution for every problem, it still needs the help of classical computing to be able to perform its more advantageous tasks. 

Classical computing is great at organising and shuffling data around and performing parallel actions on your device, but with the help of QC we would be able to shift the heavy long term calculations over to devices especially made for long term and hard calculations like a quantum processor. Calculating a machine learning model (\textcite{Schuld2014}) or performing an accurate simulation of a new medicine could be exponentially reduced in time, which would return the value of these calculations to the business side in a much faster way. \autocite{Schuld2015} \autocite{Troyer2005}

\subsection{Quantum computing and the mainframe}

First of all we need to clarify what a mainframe is and what its main use is in our current business environments. A mainframe is a type of supercomputer that is different from other supercomputers because it is not specialised in solving 1 really hard problem, like simulations or factorisation, it is specialised to have the highest possible amount of secure throughput for smaller calculations. The mainframe is widely used within the banking, production and logistical sector as it offers the most reliable way of managing your data that is generated by a certain business practice.

To clarify let us look at an example where a mainframe computer like the IBM Z15 shines. When millions of users throughout the world want to buy their flight tickets towards France around the end of April, a huge bottleneck is created at the end point of the booking system of the particular airport. A mainframe handles these types of atomic transactions quickly to make sure every single booking will come through with the correct data. If the data were to be corrupted along the way, the mainframe would be able to spot out these irregularities and discard this data so that the user receives a proper notification as soon as possible. So look at a mainframe computer as a really good processor of input and output of small tasks.

To quickly compare, a normal super computer is used because of its high-speed processors that are able to create simulations, ML-models... They are designed to perform one synchronous task really well and if necessary brute force this task.

IBM has released the new mainframe Z15 in 2019, with a broad future perspective, because as one of the top researchers in quantum technology they have a clear image of how a quantum computer could influence themselves and others within their sector.
They are emphasising on 2 very different aspects to make sure their devices are the most likely to take the biggest market share, modernisation and security. 
With modernisation IBM is trying the incorporate the mainframe in as much areas as possible to keep on attracting new developers so that their devices don't fall behind. With this modernisation a lot of opportunities are opening up to connect widely different departments such as Machine learning and quantum research.

They have also emphasized on creating new security measures which focusses more on digital signing compared to the current RSA factorisation algorithm. This could secure the mainframe security status indefinitely. Quantum would in the future indeed be able to brute force these RSA based algorithms (\textcite{Shor2000}) and that is why data-security has become such a high importance area at the moment for everyone in computer science that knows the potential hazard of powerful quantum devices.

So now that you are able to view what role the mainframe plays, we can more clearly look at how quantum computers could offer major benefits as a complementary service for solving the harder problems just like a super computer works with the mainframe in a similar way. Nowadays all the data generated from the billions of transactions from the mainframe are preserved so that afterwards a supercomputer would be able to process all this information inside a reasonable time frame to get the best possible business value out of it. If the quantum computer would be able to help process this data exponentially faster, the business value of this data would also exponentially increase. Mainframes offer this great amount of throughput of data that could offer this exponential force of QC the data-items it needs to become viable for business. Again this is something that could show how well QC can fit in our existing model of computation, to further enhance the processes that drive our economies. QC will become the most helpful ally to make sure all our classical systems become even more valuable.

Mainframes are here to stay, with the dawn of fully data-driven worlds that operate on an autonomous platform and if QC could add to such an industry important device it will most definitely boost both devices in exposure and value. But for now there most certainly are questions of when QC would be able to reliably process and/or handle the high throughput of a mainframe in a manner that would add to its business value. This will form another benchmark for QC where full cooperation of these devices can transform into something practical and valuable.




\subsection{Quantum computing and Machine Learning}

Another area where QC could have a major impact is the area of Machine Learning \textbf{(ML)}. At this moment machine learning is running into a bottleneck where the amount of data has become so intense that ordinary classical computers are not able to process the data in time so that its value can be exploited to its maximum potential. QC could help with this issue in a couple of major aspects, like data model training and data capturing. This would greatly improve the impact of ML on the business side, because the relay of the captured information through the models could indeed be shortened in exponential ways. \autocite{Biamonte2017}

At this moment research is becoming quite prevalent in ML with a combination of QC-technology. Qiskit has also seen this opportunity opening up and they too try and attract businesses with these advantages. The current research groups at IBM and Google are able to enhance supervised learning algorithms as well as unsupervised learning, with time series or without. Algorithms such as linear regression, k-means clustering and even neural networks can be enhanced during its training phases with QC. Due to superposition and entanglement, these algorithms could train a model theoretically through one loop instead of having multiple epochs that contain a certain batch size, which obviously speeds up the generation of models that require a large amount of data to become valuable.

The utilisation of QC with ML would not aid the accuracy of ML in the short run because of the uncertainty of solving the quantum decoherence issue for now. But the time frame of processing a complete machine learning model could eventually be exponentially decreased.






\input{beneficiary-sectors}
\input{creating-the-future}
%%=============================================================================
%% Methodologie
%%=============================================================================

\chapter{Practical demonstration with Qiskit}
\label{ch:practical}

For 2 decades now people have been receiving fully blown quantum mechanics courses where they are able to experiment with the mere thought of quantum experiments in a theoretical type of way, but never were truly interested parties able to perform their experiments in a free and fluid manner. QC is at a point where we are able to effectively experiment with the technology as a broader community. Platforms like Qiskit are excellent in their reach towards interested parties and are more than welcoming towards new developments that could aid the whole community in its research. The service is open source which truly pushes the whole movement of research out of this shroud of high costs and large enterprises. This will obviously influence other branches to follow in the same footsteps as to allow every party that is interested or has a passion to be able to participate in a costless and open manner. To remain objective and fair towards other companies outside of IBM, Google is also participating in the open source community with platforms like Cirq, \textcite{Cirq}. 

In the following part, we will lay out how interested parties are able to perform their own executions on real devices and start applying what some of them have been learning theoretically for over 20 years. Whilst trying to resolve the main question of this specific paper `How will quantum computing affect the mainframe environment and its applications? `, a major roadblock has been discovered, which is interesting none the less because it shows were research and engineering has not yet ventured far enough to overcome them. 


\subsection{Grover's search algorithm in a practical fashion}

Let us start of practical with one of the two most well known quantum algorithms, which is the Grover Search algorithm. If this algorithm were to be applicable on a large scale it could indeed affect the current mainframe environment of DB2 databases in a drastic approach. The whole premise of the algorithm is that we are able to speed up the search time in an unstructured database quadratically. This all meaning when a computer needs to find an item with an unique attribute that differentiates itself from the other items in the list, QC could become the main solution. The whole algorithm uses something called "amplitude amplification" where the algorithm influences the probabilities in such a manner that the correct item has the highest probability after the Quantum computation. \autocite{Grover1996}

For the experiment itself, we have chosen for the "Boolean satisfiability problem" which uses Grover's way of amplitude amplification to find the correct result. This computer science question goes as follows, given a boolean comparison of multiple parts are we able to determine the outcome to get that specific TRUE. Being to solve this comparison in a way that could abuse the fact of superposition could prove useful when we scale out the problem towards thousands or even millions of comparisons for other algorithms. For now the 3-SAT problem has been chosen to be performed using Qiskit to show off the potential of QC for now.

You are able to view the function stated below as the problem that we will try and solve using Quantum technology. The algorithm now needs to find which solutions are possible by interchanging x,y,z with TRUE/FALSE.

$ f(x,y,z) = (\neg x \vee y \vee \neg z) \wedge  ( x \vee \neg y \vee \neg z) \wedge ( x \vee \neg y \vee  z) \wedge (\neg x \vee \neg y \vee z) \wedge  ( x \vee y \vee  z)	 $
				 
Using a simulation of a quantum computer we are able to show the results in figure below. The probabilities have been amplified to where they are the correct results of this boolean expression. The first figure below show the probabilities through the simulation, meaning that these probabilities have not been influenced by quantum decoherence on the real device it is more clearly visible that without any form of quantum error correction QC will run into a brick wall. 

\begin{figure}[h]
	\centering
	\includegraphics[scale = 0.75]{../Demonstration/img/simulated_3SAT.PNG}
	\caption{These are the results of executing the algorithm for the 3-SAT problem on a \textbf{quantum simulator} that comes with Qiskit. The encoding refers to the TRUE/FALSE value of the x,y,z respectively}
\end{figure}

If you feel that your algorithm is on point by testing it through the simulator, IBM allows its recreational users to send off their circuits to real devices that have functional qubits to play around with. ( At the time of writing the ibmq16Melbourne device had 15 qubits available to mess around with)

The reason for choosing this specific experiment is to show that even problems that just require us to encode boolean statements we needed 694 quantum gates. IBMQ transpiles the sent-off circuit to the necessary amount of gates needed for this specific calculation. It does not keep in account that having this much gates on a single line of computation invites a multitude of quantum decoherence issues during runtime.

\begin{figure}[h]
	\centering
	\includegraphics[scale = 0.75]{../Demonstration/img/real_device_3SAT.PNG}
	\caption{These are the results of executing the algorithm for the 3-SAT problem using 15 qubits on a \textbf{real quantum device} that comes with Qiskit. The encoding refers to the TRUE/FALSE value of the x,y,z respectively}
\end{figure}

If you have looked closely at the image above (4.2) you are able to see that decoherence for now breaks the probabilities of a computation too much to reliably trust any computation of this size out of a quantum computer. The values become distorted over time by all types of interference even if all the interference from inside the machine is not accounted for the machine can become influenced by a single external interaction like temperature, pressure etc. 



If we compare the manually gathered results in the table above across the simulated version and the real version, we are clearly able to see that decoherence has played too big of a role to be certain of any output for these types of large computations.

Let us work out a clear example to make sure our probabilities are incorrect. If we take the highest probability of the real execution which is the configuration of $101$. Meaning that the quantum computer determined that when X and Z are true the whole boolean expression will result in a returned value of TRUE. This is simply not a valid option for this boolean expression. If we look at the first part of this boolean expression we can see that this configuration would return a FALSE resulting in the whole expression being FALSE because all the parts are connected with a logical AND. 

$f(1,0,1) = (0 \vee 0 \vee 0 \vee) \wedge (1 \vee 1 \vee 0)  \wedge ( 1 \vee 1 \vee  1) \wedge ( 0 \vee 1 \vee 1) \wedge  ( 1 \vee 0 \vee  1)$

 

As for now we are able to play around with the greater problems of quantum computing but to be able to reliably solve real-world solutions in a beneficial way remains an uncertainty.

So with the current state of engineering, computer scientists will have to wait to fully utilise the system in a reliable fashion. But as engineering develops the power of quantum computing will increase exponentially with each added qubit to the system, which would make algorithms like this extremely valuable for data-crunching. When we find a way to circumvent the interference of quantum decoherence or when we reliably fix the errors it produces, a quantum system could become an essential tool for every sector willing to innovate. 



\subsubsection{Data-encoding in QC}

As the experiment clearly shows there is an actual advantage reachable with quantum computing. Of course quantum decoherence is a main aspect of QC and solving it would be greatly beneficial for the whole sector, but there is also a different problem that arises with defining our classical way of problems in a quantum way. The way we represent data in a quantum circuit quickly becomes overly complicated for any large database structure. Quantum computers are great in predicting what quantum effects will occur and where quantum physics influences specific sectors. The issue lies in the fact that we want to input our classical database into a quantum device and hopefully receive the results in a readable classical solution. 

This shows an issue we are facing with the encoding of our classical data to quantum data and back. For the proof-of-concept experiments it does not matter as the encoding time really does not influence the experiment as a whole. But once we start scaling out the issue where we would want to find a specific item through the use of Grover's algorithm, we would run into the issue that the encoding and decoding of the input and output could take up a great amount of computational time. If however QC develops in such a way that we are able to gain the full benefits of qubits in superposition this encoding time could be overcome, but for now it remains a crucial factor in solving the whole feasibility of QC.



\subsection{Mainframe computing with QC}

As the paper has previously stated having QC together with the power of a mainframe could become extremely advantageous for the whole industry to provide the power of data crunching this immense layer of internal data that companies have collected over the years. So we needed to find a circuit that could show of where quantum computing indeed could benefit in the crunching of data in a better/ faster way than classical computing can at the moment. Soon it became clear that simulating anything of a mainframe is impossible for now, we can simulate how a new form of database search could work with Grover. But we are not able to simulate the main advantage of a mainframe device, which is performing quick, stable and secure input and output transformations. And as shown by the experiment it is obvious that having a stable output of a specific input is not one of the main strengths of QC. Then when we take into account the encoding and decoding of classical computations and problems, which would greatly slow down the performance of a mainframe.

So for now there is no clear advantage when we use the current developments of QC with the existing mainframe technology. It does not take away its immense potential when QC is able to process the complete I/O of a mainframe in an exponentially smaller time frame than classical computing processing is able to do now.

Then obviously there remains the issue that quantum processors don't have the capability to actually perform algorithms that require a greater amount of qubits due to decoherence and previously stated problems.




% Voeg hier je eigen hoofdstukken toe die de ``corpus'' van je bachelorproef
% vormen. De structuur en titels hangen af van je eigen onderzoek. Je kan bv.
% elke fase in je onderzoek in een apart hoofdstuk bespreken.

%\input{...}
%\input{...}
%...

%%=============================================================================
%% Conclusie
%%=============================================================================

\chapter{Discussion}
\label{ch:conclusie}

Quantum computing will most definitely become one of the great buzzwords of the next decade. With the release of \textcite{Google2019} around their interpretation of 'Quantum supremacy', the whole field was catapulted to the forefront of research. All big players in Quantum research have been given a tremendous spotlight for the future of profitable quantum computing. And this is precisely why we need to make the concept more approachable for anybody interested.

\subsection{Quantum computing for now}

Firstly, after thorough research, a couple of interesting conclusions can be drawn. Beginning with one of the most important ones, Quantum computing is here to stay. The technology has given too much promise in too many sectors that have a strong financial backbone. At this point it is important to understand that the world around us that is visible with the naked eye is completely homogeneous with the 'quantum world', meaning that all research to exploring our world around us, can only prove to be profitable in the future.

Secondly returning to the more concrete conclusions of this paper. While executing this specific version of Grover's algorithm, there was a clear trend visible between the theory and the practical example that does show there are some growing pains that come with the expansion of our quantum devices.
Yes, the theory can be fully implemented at this point to visualise its results when we would work in the most perfect of environments like a simulator. But this perfect environment simply does not exist. Meaning that we will have to be creative to reach this edge of perfect conditions to get to a point that these simulated algorithms can become reliable and profitable towards the future. There are two ways of trying to fix the issue of quantum decoherence. Firstly we could create devices that are not influenced by any internal/ external interferences. They could provide a stable platform for algorithms like Shor's encryption breaking algorithm \textcite{gidney2019factor}. The other option would be to account for these interferences to happen anyway and try and compensate them in a software way, much in the same way a computer does error correction for downloaded files. The latter seems to be the more reasonable option where we are already trying to implement these quantum error correction in to provide better results \autocite{Cory1998}. For now there is no clear technique behind the whole principle except to play around with the length that a qubit needs to stay in its elevated $\ket{1}$ state and the length of the circuit as a whole

But this can all be tried out. Because at the moment we are able to extensively experiment with real or simulated quantum devices. There is a multitude of platforms available, most of them are open-source and free to contribute to them as you want to expand their feature-set. Frameworks like Qiskit allow users to design quantum circuits and test them using their built in simulators, which are easy to pick up but hard to master. As of now IBMQ is the only service that allows you to push up your circuits to really test them on a real device owned and managed by IBM. By granting people the privilege to experiment around with the real devices and notice the shortcomings through raw data results, really shows off how dedicated the whole community of computer science is on pushing this technology to the forefront.


\subsection{Quantum computing and its myths}

As shown in the experiment, QC will not change our entire world in the next year in any drastic manner. So theories that quantum computing could break our entire encryption standard in a matter of years seem absurd once you look at the real executions of these needed algorithms on real quantum devices.
Nevertheless we do want to work proactively to have solutions ready-to-go once these machines do become powerful enough to brute force the RSA-encryption by finding the factors of the prime numbers that represent the private keys of encrypted files.
Some researchers also believe that quantum decoherence will prove to be an unpassable obstacle the more qubits we start adding to the systems. This obstacle is the fact that adding more qubits to a system will always increase the internal interference exponentially and thus generate too much decoherence. This indeed is a major hurdle that needs to be overcome to make quantum computing the new standard for solving really hard to solve problems in the computer science community.

\subsection{Quantum computing as an addition}

The one thing we should take away from the dawn of quantum computing the following decade, is that quantum computing is not the one solution for every single issue in computing. It has its advantages and disadvantages just like classical computing. We should strive to make these two technologies as complementary as possible so that they can cancel out each others disadvantages and amplify their advantages. 

With this work, we have tried to inspire people to learn more about the subject of quantum computing and to hopefully entice them into writing their own 'Hello-world-Applications' on any of the freely available frameworks. For the actual executions and setup of the used experiments, we refer to appendix A.

\subsection{Future works}

The potential powerhouse of a mainframe device and a quantum computer can still prove to be advantageous in future works. If IBM keeps up with doubling its pace of releasing quantum computational resources into the world each year, there may be a chance that mainframes and QC prove to be viable in the near future. Also quantum error correction needs to evolve to an acceptable percentage to where we could start thinking about implementing QC machine learning algorithms in our mainframes or even the speeding up the database structures in the devices by using algorithms like Grover's. But for now QC and mainframe are not able to cooperate in an useful manner to add more business value. 

It is useful to explore these options upfront so that we know when the time comes to expand QC towards the mainframe we have a clear scope to all the potential applications of these two devices.


\begin{figure}[h]
	\centering
	\includegraphics[scale = 0.75]{../Demonstration/img/qiskit_logo.PNG}
	\caption{The platform we used for executing the 3-SAT algorithm on a real quantum device. © Copyright 2020, Qiskit Development Team Last updated on 2020/05/14.}
\end{figure}




%%=============================================================================
%% Bijlagen
%%=============================================================================

\appendix
\renewcommand{\chaptername}{Appendix}

%%---------- Onderzoeksvoorstel -----------------------------------------------

\chapter{Research Proposition}

Under this section you are able to view the original proposition for this paper to introduce the subject with schooling officials and technical promotors. 
This section can also serve as an introduction to this paper for any further interested readers.

To address the whole reason why this paper was created, the subject has become more and more influential in the Computer Science world.
We have officialy come at a point where we are able to think of real world utilisations of quantum computers to further our research in various subjects.
Quantum has become a buzz word at this point, but not everyone that throws it around has a real grasp on what it exactly means.
That is why this paper has been created to aid interested people in the subject to gain a real understanding of what quantum actually is and what it can do.




% Verwijzing naar het bestand met de inhoud van het onderzoeksvoorstel
%---------- Inleiding ---------------------------------------------------------

\section{Introduction} % The \section*{} command stops section numbering
\label{sec:introductie}

With the approaching realisations of quantum technology, the interest in the subject has risen exponentially. There has been a strong believe in the last 30 years that quantum computing can and will influence our environment more than we think. The mainframe environment is one of the sectors that can become the most influential in \emph{computer science}, because of its immense creation of data. Data will become the driving factor inside our societies, think of how much our daily lives are controlled by data ( e.g. online shopping, social media etc.). With the usage of mainframes we are able to create a sense of logic in this almost infinite pile of data. But through the utilisation of quantum computing, data exploration and mining can become much more thorough and meaningful for business applications. The main driving factors for technological breakthroughs have always been wars and economics ( e.g. Atomic energy, commercial aircraft, radio etc.). The more applications of quantum computing that we are able to find for existing economical applications, the more general investment in research will be made. Which would obviously boost both fields at once. In this paper we will try and find these general applications in reality of quantum computing.

\subsection{Topics}
\begin{itemize}
  \item Security implications with the rise of quantum computing
  \item Efficiently exploring mainframe data using quantum computing
  \item Advantages and disadvantages of combining classical computing with quantum computing
  \item Building quantum software before the creation of the hardware
\end{itemize}

%---------- Stand van zaken ---------------------------------------------------

\section{State-of-the-art}
\label{sec:state-of-the-art}
\subsection{Prior knowledge}
Inside the paper a couple of physics associated terms will be utilised. If you are not familiar with basic quantum physics notations, it would be highly recommended to read one or both of the following papers, ~\textcite{Rieffel1998} or ~\textcite{Shor2000}. It is also possible to read this paper as an informational piece without the implications of the mathematics and physics surrounding the subject. As previously stated the paper will not be going in depth technologically, because the paper wants to expose the practical usages of quantum computing compared to classical computing and because that would reach far out of the scope of this paper.

\subsection{Literature review}

As of now Google has claimed to have won the \emph{Quantum Supremacy race} ~\autocite{Google2019}
% Voor literatuurverwijzingen zijn er twee belangrijke 

Je mag gerust gebruik maken van subsecties in dit onderdeel.

%---------- Methodologie ------------------------------------------------------
\section{Methodology}
\label{sec:methodologie}

Hier beschrijf je hoe je van plan bent het onderzoek te voeren. Welke onderzoekstechniek ga je toepassen om elk van je onderzoeksvragen te beantwoorden? Gebruik je hiervoor experimenten, vragenlijsten, simulaties? Je beschrijft ook al welke tools je denkt hiervoor te gebruiken of te ontwikkelen.

%---------- Verwachte resultaten ----------------------------------------------
\section{Expected results}
\label{sec:verwachte_resultaten}

Hier beschrijf je welke resultaten je verwacht. Als je metingen en simulaties uitvoert, kan je hier al mock-ups maken van de grafieken samen met de verwachte conclusies. Benoem zeker al je assen en de stukken van de grafiek die je gaat gebruiken. Dit zorgt ervoor dat je concreet weet hoe je je data gaat moeten structureren.

%---------- Verwachte conclusies ----------------------------------------------
\section{Expected conclusions}
\label{sec:verwachte_conclusies}

Hier beschrijf je wat je verwacht uit je onderzoek, met de motivatie waarom. Het is \textbf{niet} erg indien uit je onderzoek andere resultaten en conclusies vloeien dan dat je hier beschrijft: het is dan juist interessant om te onderzoeken waarom jouw hypothesen niet overeenkomen met de resultaten.



%%---------- Andere bijlagen --------------------------------------------------
% TODO: Voeg hier eventuele andere bijlagen toe
%\input{...}

%%---------- Referentielijst --------------------------------------------------

\printbibliography[heading=bibintoc]

\end{document}
