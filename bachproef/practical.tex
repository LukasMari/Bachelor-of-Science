%%=============================================================================
%% Methodologie
%%=============================================================================

\chapter{Practical demonstration with Qiskit}
\label{ch:practical}

For two decades now people have been receiving extensive, theoretical quantum mechanics courses where they are able to experiment with the thought of quantum experiments in a theoretical type of way, but were never truly interested parties able to perform their experiments in a free and fluid manner. QC is at a point where we are able to effectively experiment with the technology as a broader community. Platforms like Qiskit are excellent in their reach towards interested parties and are more than welcoming towards new developments that could aid the whole community in its research and adaption. The service is open source which truly pushes the whole movement of research out of this shroud of high costs and large enterprises. This will obviously influence other branches to follow in the same footsteps to allow every party that is interested or has a passion to participate in a costless and open manner. To remain objective and fair towards other companies outside of IBM, Google is also participating in the open source community with platforms like Cirq. (\textcite{Cirq}). 

In the following part, we will lay out how any interested party is able to perform their own executions on real devices and start applying what some of them have been learning theoretically for over 20 years.

\subsection{Grover's search algorithm in a practical fashion}
\subsubsection{Grover's search unstructured database search}

We have chosen for an adaptation of an algorithm that could prove extremely useful for any implementation combined with a mainframe, for example the Grover Search algorithm applied for the boolean satisfiability problem. The whole premise of the original algorithm is that we are able to speed up the search time in an unstructured database. This all meaning when a computer needs to find an item with a unique attribute that differentiates itself from the other items in the list, QC could become the main solution to a brute force task like that. The whole algorithm uses "amplitude amplification" where the algorithm influences the probabilities in such a manner that the specific item has the highest probability after the quantum computation. \autocite{Grover1996}

To clarify amplitude amplification, this is a principle that affects qubits in their superposition and entangled states where they would interfere with each other to eliminate the least likely outcomes and amplify the statistical chance of the desired item. By applying this to an algorithm after the correct transformations, you would be able to decompose all the different results back to a highly probable result that does not collapse all the qubits in superposition like any normal measurement would. 

\subsubsection{Grover's search algorithm in an applied form}

For the experiment itself, we have chosen a specific "Boolean satisfiability problem" which uses Grover's way of amplitude amplification to find the correct results of a boolean problem. This computer science question goes as follows, given a boolean comparison of multiple parts are we able to determine the outcome of this function to where the result of the boolean calculation equal TRUE. Being able to solve this comparison in a single execution could abuse the fact of superposition and entanglement and could prove useful when we scale out the problem towards thousands or even millions of factors for other functions. For now the 3-SAT problem has been chosen to be performed using Qiskit to show off the performance of QC that is available to us.

You are able to view the boolean function in figure 4.1 as the problem that we will try to solve using Quantum technology. The algorithm now needs to find which solutions are possible by interchanging x,y,z with TRUE/FALSE.

\begin{figure}
	$ f(x,y,z) = (\neg x \vee y \vee \neg z) \wedge  ( x \vee \neg y \vee \neg z) \wedge ( x \vee \neg y \vee  z) \wedge (\neg x \vee \neg y \vee z) \wedge  ( x \vee y \vee  z)	 $
	\caption{The 3-SAT problem that we have processed manually, using the quantum simulator and using the real quantum device.}
\end{figure}

				 
\subsubsection{Executing the quantum algorithm}				 

Using a simulator of an ideal quantum computer we are able to show the results in figure 4.2. The probabilities have been amplified to where there are multiple results for this boolean expression. Figure 4.3 shows the gathered results when we sent off the identical circuit towards one of IBM's real quantum devices (IBMQMelbourne16). 

\begin{figure}[h]
	\centering
	\includegraphics[scale = 0.75]{../Demonstration/img/simulated_3SAT.PNG}
	\caption{These are the results of executing the algorithm for the 3-SAT problem on a \textbf{quantum simulator} that comes with Qiskit. The encoding refers to the TRUE/FALSE value of the x,y,z respectively}
\end{figure}


\begin{figure}[h]
	\centering
	\includegraphics[scale = 0.75]{../Demonstration/img/real_device_3SAT.PNG}
	\caption{These are the results of executing the algorithm for the 3-SAT problem using 15 qubits on a real quantum device that comes with Qiskit. The encoding refers to the TRUE/FALSE value of the x,y,z respectively}
\end{figure}

The reason this specific experiment is used, is to show that even for this problem of solving boolean expressions 694 quantum gates were needed. IBMQ transpiles the sent-off circuit to the necessary amount of gates needed for this specific calculation. (Appendix A)

Comparing figures 4.2 and 4.3, you are able to see that decoherence for now breaks the probabilities of a computation too much to reliably trust any computation of this size out of a quantum computer. The values became distorted over time by all types of interference.

\begin{table}
	\centering
	\begin{tabular}{|| c c ||}
		\hline
		$f(x,y,z)$ & TRUE/FALSE \\
		\hline\hline
		0 0 0 & FALSE \\ 
		0 0 1 & TRUE \\
		0 1 0 & FALSE \\
		0 1 1 & FALSE \\
		1 0 0 & TRUE \\
		1 0 1 & FALSE \\
		1 1 0 & FALSE \\
		1 1 1 & TRUE \\
		\hline
	\end{tabular}
\caption{These are the results of our 3-SAT problem, calculated in a manual classical manner.}
\end{table}

If we compare the manually gathered results from table 4.1 above across the simulated version and the real version, we are clearly able to see that decoherence has played too big of a role to be certain of any reliable output for these types of large computations. By having these big differences in results we prove that simply adding qubits will not be helpful for the reliability in QC. The results differ too much to use this circuit in any practical business cases where speed is as critical as accuracy in reporting towards external parties.

Let us work out a clear example (figure 4.4) to make sure our probabilities generated by the real quantum device are incorrect. If we take the highest probability of the real execution which is the configuration of $101$. Meaning that the quantum computer determined that when X and Z are true the whole boolean expression will result in a returned value of TRUE. This is simply not a valid option for this boolean expression. If we look at the first part of this boolean expression we can see that this configuration would return a FALSE resulting in the whole expression being FALSE because all the parts are connected with a logical AND. 

\begin{figure}
	\centering
	$f(1,0,1) = (0 \vee 0 \vee 0 \vee) \wedge (1 \vee 1 \vee 0)  \wedge ( 1 \vee 1 \vee  1) \wedge ( 0 \vee 1 \vee 1) \wedge  ( 1 \vee 0 \vee  1)$
	\caption{The filled-in example for showing how performing this boolean equation on the real device has been influenced by quantum decoherence.}
\end{figure}



\subsection{Data-encoding in QC}

As the experiment shows we are able to use quantum algorithms to solve classical computer problems. Of course quantum decoherence has played a great role in computation and solving it would be greatly beneficial for the whole sector, but there is also a different problem that arises with defining our problems in a quantum way. The way we represent data in a quantum circuit quickly becomes too complex for any large database structure. This is visible from appendix A, where you are able to see how many gates were needed to perform this rather small boolean equation. (694 gates)

So for now quantum computers are not as powerful as classical machines in solving classical problems. But they can have value in sectors that are most halted by the use of classical devices. E.g. using them for use-cases like simulating quantum effects we can easily imagine that using a qubit to actually simulate quantum effects is way more effective than simulating a qubit using classical bits.

The issue lies in the fact that we want to input classical database structures/ data-items into a quantum device and hopefully receive the results in a readable classical solution. So if we want to solve existing classical problems we need to find a performant manner of encoding these problems in a quantum device. As appendix A shows, we would need 694 quantum gates to encode and solve this small boolean problem. The generation and application of these gates itself brings a lot of potential decoherence and risk in the reliability of the results, which in turn downgrades the experience of working with the quantum devices as of now.

All of this shows an issue we are facing with the encoding of our classical data to quantum data and back. For the proof-of-concept experiments it does not matter as the encoding time does not influence the runtime of the experiment as a whole. On a normal classical machine, it took 2 seconds to solve this boolean expression and on the quantum device the execution took 7.3 seconds. But once we start scaling out the issue wanting to find a specific item through the use of Grover's algorithm, we would run into the issue that the encoding and decoding of the input and output could take up a great amount of computational time. If however QC develops in such a way that we are able to gain the full benefits of qubits in superposition, this encoding time could be overcome. But for now it remains a crucial factor in solving the feasibility of QC.



\subsection{Mainframe computing with QC}

As the paper has previously stated having QC together with the power of a mainframe could become extremely advantageous for the whole industry to provide the power of data crunching this immense layer of internal data that companies have collected over the years with their mainframes. So we needed to find a circuit that could show off where quantum computing indeed could benefit in the crunching of data in a better/ faster way than classical computing at the moment could. Soon it became clear that simulating anything of a mainframe is impossible for now, we can simulate how a new form of database search could work with Grover. But we are not able to simulate the main advantage of a mainframe device, to perform quick, stable and secure input and output transformations. And as shown by the experiment it is obvious that having a stable output of a specific input is not one of the main strengths of QC for now. Then when we take into account the encoding and decoding of classical computations and problems it only decreased the feasibility for running mainframe-sized data-sets.


So for now there is no clear advantage when we would use the current developments of QC with the existing mainframe technology, because as stated above we ran into the issue of data-encoding. The potential for QC and mainframe to still become the data-crunching powerhouse of the future is still there. But for this potential to become viable, a couple of major hurdles have to be overcome first.


\subsection{Future prospects}

Now we are able to examine the greater problems of quantum computing but to be able to \textit{reliably} solve real-world solutions in a beneficial way remains an uncertainty.

With the current state of engineering, computer scientists will have to wait to fully utilise the system in a reliable fashion. But as engineering develops the power of quantum computing will increase exponentially as stated by Neven's law \autocite{Hartnett2019} with each added qubit to the system, which would make algorithms like this extremely valuable for data-crunching. When we find a way to circumvent the interference of quantum decoherence and the data-encoding issue, a quantum system could become an essential tool for every sector willing to innovate. 


