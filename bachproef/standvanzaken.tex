\chapter{\IfLanguageName{dutch}{Stand van zaken}{State of the art}}
\label{ch:stand-van-zaken}

% Tip: Begin elk hoofdstuk met een paragraaf inleiding die beschrijft hoe
% dit hoofdstuk past binnen het geheel van de bachelorproef. Geef in het
% bijzonder aan wat de link is met het vorige en volgende hoofdstuk.

% Pas na deze inleidende paragraaf komt de eerste sectiehoofding.

\subsection{General explication of quantum computing}

There are a few advantages that quantum computers have over any classical machine, to understand them we will have to introduce a couple of explanations.

The first is \emph{superposition}, it is the term used to explain how a quantum bit ( Qubit) can be in multiple states at once. In classical computing we are able to represent 1 state at a time with the use of a normal bit, 0 or 1. In quantum computing this is different through the use of superposition, a qubit can be in both states at one point in computation. ~\autocite{Shor2000} Effectively this means that quantum computers can exponentially gain computational space through only the addition of 1 qubit, while a classical computer would add bits to only linearly gain computational space. But this only counts inside the computation, because as we know a qubit will 'fall' in a single state as soon as it is observed/ measured. \autocite{Rieffel1998}

A practical example for a clearer understanding of the term. Suppose that we have a classical system where we have access to 4 bits of processing space, this would effectively mean that we are able to represent 16 different states. If we compare it now we suppose a quantum system that utilizes 4 qubits of processing space, if we take in accounts the use of superposition 1 qubit is able to represent 2 classical bits. Meaning that we now have access to 16 computational space in classical bits, which also means that we have 65536 different states to represent. 

\begin{equation}
Classical:\quad 4 \quad bits => 2^{4} states
\end{equation}
\begin{equation}
Quantum: \quad 4 \quad qubits => 16^{4} states
\end{equation}

Another really powerful tool for quantum computations is called \emph{entanglement}. It describes the physical phenomenon that 2 particles can become entangled, meaning that ones state can directly influence the state of the other over an infinite distance. In our case, the particles represent the qubits that can become entangle which could mean that if 1 qubit is measured in a specific state and it is entangled with another qubit, it means that we are able to know the state of the other qubit \emph{without} measuring it. Think of the possible utilisation in communication in between devices and/ or networks that this physical phenomenon could introduce.

\emph{Decoherence} is also a term that forms a big issue right now for further advancements in quantum computing. Decoherence is the term that describes how a qubit can lose its quantum capabilities over time or due to interference from the outside world, currently to achieve quantum aspects we need to cool down the quantum devices to around 0 Kelvin or -273.15 degrees Celsius. Only slight fluctuations of the near perfect conditions can mean that the qubits lose their quantum capabilities, which would mean that the eventual computation becomes useless. As of now the Sycamore processor from Google has achieved the largest amount of qubits (53) to be used for a computation with 'fighting off' the decoherence. With 'fighting off' we are referring towards the process of Quantum Error Correction ( QEC), this is a process that tries to prevent the decoherence and interference between the qubits. This is as of now the main restriction on just expanding the number of qubits, because the more qubits are used in the system the more the system will be susceptible for decoherence and interference. ~\autocite{Cory1998}

\subsection{Practical fields of study using quantum}

\subsubsection{Security}
A great concern with the rise of quantum computing has come up, people are believing that our current encryption system is easily breached by quantum computation. But indeed everyone was shocked when Peter Shor released his paper ~\textcite{Shor1994}, which described a way to resolve prime factorization on quantum computers. A classical computer can easily find the result of a multiplication but finding the factors of a large number is an exponential computation for a normal computer. Shor's algorithm uses superposition and quantum Fourrier transform to resolve this exponential issue, meaning that it would suddenly become really easy to brute-force encryptions of data ( e.g. RSA).~\autocite{Rivest1978} This could possibly mean that the praised mainframe environment would become a lot less secure. In reality however factorization of large numbers is not the only way that we encrypt our data. For example the AES algorithm uses multiple partitions of a key to encrypt its data, which would re-introduce the issue of exponential computation even when using a quantum computer.

~\autocite{Daemen2000} ~\autocite{IBM2019}

In effect the introduction of quantum computing will most likely result in a much safer environment, because through the utilization of quantum encryption our data could even become impenetrable for quantum computers. When quantum computing will be implemented alongside the already existing mainframe applications, it will only strengthen its safety aspects and also the speed of these encryptions in mass.
\subsubsection{Data}
\subsubsection{Physics and chemistry}
\subsubsection{Open source software}


\lipsum[7-20]
