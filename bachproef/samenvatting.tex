%%=============================================================================
%% Samenvatting
%%=============================================================================

% TODO: De "abstract" of samenvatting is een kernachtige (~ 1 blz. voor een
% thesis) synthese van het document.
%
% Deze aspecten moeten zeker aan bod komen:
% - Context: waarom is dit werk belangrijk?
% - Nood: waarom moest dit onderzocht worden?
% - Taak: wat heb je precies gedaan?
% - Object: wat staat in dit document geschreven?
% - Resultaat: wat was het resultaat?
% - Conclusie: wat is/zijn de belangrijkste conclusie(s)?
% - Perspectief: blijven er nog vragen open die in de toekomst nog kunnen
%    onderzocht worden? Wat is een mogelijk vervolg voor jouw onderzoek?
%
% LET OP! Een samenvatting is GEEN voorwoord!

%%---------- Nederlandse samenvatting -----------------------------------------
%
% TODO: Als je je bachelorproef in het Engels schrijft, moet je eerst een
% Nederlandse samenvatting invoegen. Haal daarvoor onderstaande code uit
% commentaar.
% Wie zijn bachelorproef in het Nederlands schrijft, kan dit negeren, de inhoud
% wordt niet in het document ingevoegd.

\IfLanguageName{english}{
\selectlanguage{dutch}
\chapter*{Samenvatting}

Zoals het onderzoek aantoont is het onderzoeksgebied van kwantum computers nog in zijn beginjaren en moeten we kritisch blijven ten opzichte van elke nieuwe uitgave in verband met nieuw onderzoek. Echter is het ook belangrijk dat we buiten het puur theoretische deel ook effectief op zoek gaan naar de praktische toepassingen en/ of inzichten in onze huidige processen met evt. de toepassing van kwantum verwerking. 

In het onderzoek proberen we een duidelijk beeld weer te geven aan de lezer, zodat hij/ zij zelfstandig kan nadenken over toepassingen en/ of zelf toevoegingen maken aan de vele open source gemeenschappen op Github. Dit hebben we proberen te bereiken door enkele praktische vergelijkingen te maken met de hulp van het Python-framework Qiskit tussen de uitvoering op een klassiek systeem en een kwantum systeem. De resultaten wijzen inderdaad op een mogelijke versnelling, maar zoals te zien aan de werkelijke uitvoeringen op de echte kwantum computers van IBM is het moeilijk om deze nieuwe technologieën al meteen te gebruiken in bestaande productiesystemen. 

We proberen ook alle uitkomsten te relativeren en ervoor te zorgen dat de lezer volledig op de hoogte is van de potentiële valkuilen met kwantum computers en hun toepassingen. 


\selectlanguage{english}
}{}

%%---------- Samenvatting -----------------------------------------------------
% De samenvatting in de hoofdtaal van het document

\chapter*{\IfLanguageName{dutch}{Samenvatting}{Abstract}}

As the research shows, the field of quantum computers is still in its early years and we must remain critical of any new publication related to new research. However, it is also important that beyond the purely theoretical part, we also effectively look for practical applications and/or insights into our current processes with the possible application of quantum processing. 

In the research we try to present a more clear picture to the reader, so that he or she can independently think about applications and/or make additions to the many open source communities on Github. We have tried to achieve this by making some practical comparisons with the help of the Python framework Qiskit between classical computing and quantum computing. The results indeed point to  possible acceleration, but as can be seen from the actual executions on IBM's real quantum computers, it is difficult to use these new technologies in existing production systems at this point in time. 

We also try to put all the results into perspective and make sure that the reader is fully aware of the potential pitfalls with quantum computers and their applications. 

