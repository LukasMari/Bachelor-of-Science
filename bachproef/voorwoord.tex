%%=============================================================================
%% Voorwoord
%%=============================================================================

\chapter*{\IfLanguageName{dutch}{Woord vooraf}{Preface}}
\label{ch:voorwoord}

%% TODO:
%% Het voorwoord is het enige deel van de bachelorproef waar je vanuit je
%% eigen standpunt (``ik-vorm'') mag schrijven. Je kan hier bv. motiveren
%% waarom jij het onderwerp wil bespreken.
%% Vergeet ook niet te bedanken wie je geholpen/gesteund/... heeft

Why did I choose to explore the quantum realm without any specific pre-knowledge/ education? Quantum computing is being transformed to a real buzzword much like data science used to be. The field has been opened up from just highly specialised academics to an open source community willing to teach outsiders from the very basics.

Also the mere fact that the field of quantum computing is developing to a profitable and sustainable business so rapidly has astonished me from my very first contact with the environment.

 I would like to thank Frank Harkins from IBM for always being available to have a discussion about quantum computing and how it will influence our societies and even our very nature of problem-solving. 
 
 But most of all I would like to congratulate the research environment around quantum computing on how accepting and supportive they are in all interested parties. In the next decade this will only become more apparent and obvious to a point where quantum computing becomes an essential part of solving anything in a reasonable time scheme.
 
 So that is exactly why this paper will serve as a great starting tool for a computer scientist that is interested in the multiple ways Quantum Computers could influence the general sector.
