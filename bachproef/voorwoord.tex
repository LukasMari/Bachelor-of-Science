%%=============================================================================
%% Voorwoord
%%=============================================================================

\chapter*{\IfLanguageName{dutch}{Woord vooraf}{Preface}}
\label{ch:voorwoord}

%% TODO:
%% Het voorwoord is het enige deel van de bachelorproef waar je vanuit je
%% eigen standpunt (``ik-vorm'') mag schrijven. Je kan hier bv. motiveren
%% waarom jij het onderwerp wil bespreken.
%% Vergeet ook niet te bedanken wie je geholpen/gesteund/... heeft

Why did I choose to explore the quantum realm without any specific pre-knowledge/ education? Quantum is being transformed to a real buzzword much like data science used to be. The field has been opened up from highly specialised academics to an open source community willing to teach outsiders from the very start.

Also the mere fact that the field of quantum computing is developing to a profitable and sustainable business models so rapidly has astonished me from my very first contacts with the environment.

 I would like to thank Frank Harkins from IBM for always being available to have a discussion about quantum computing and how it will influence our societies and even our very nature of problem-solvings. 
 
 But most of all I would like to congratulate the research environment as a around quantum computing on how accepting and supportive they are in all interested parties. It has become as comparable to learning a new sector inside computer science. In the next decade this will only further increase to where quantum computing becomes an essential part of solving anything in a fair time scheme.
 
 So that is exactly why this paper should serve as the starting tool for computer scientist willing to expand their skill sets far outside the skill sets that is to be expected of a single computer scientist. 
