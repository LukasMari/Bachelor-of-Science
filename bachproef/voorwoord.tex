%%=============================================================================
%% Voorwoord
%%=============================================================================

\chapter*{\IfLanguageName{dutch}{Woord vooraf}{Preface}}
\label{ch:voorwoord}

%% TODO:
%% Het voorwoord is het enige deel van de bachelorproef waar je vanuit je
%% eigen standpunt (``ik-vorm'') mag schrijven. Je kan hier bv. motiveren
%% waarom jij het onderwerp wil bespreken.
%% Vergeet ook niet te bedanken wie je geholpen/gesteund/... heeft

There is a certain attraction that comes with learning new things without any proper pre-knowledge and that is precisely the reason that quantum computing has attracted me this much. 

The mere fact that the field of quantum computing is developing to a profitable and useful field so rapidly has astonished me from my very first contacts with the environment.

 I would like to thank Frank Harkins from IBM for always being available to have a discussion about the subject and its influences on our societies. But most of all I would like to congratulate the complete research environment around quantum computing on how open and supportive it has been with new developments towards any interested parties, through the  open-source communities and the many freely available papers.

